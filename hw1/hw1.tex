\documentclass{article}

    \usepackage{fancyhdr}
    \usepackage{extramarks}
    \usepackage{amsmath}
    \usepackage{amsthm}
    \usepackage{amsfonts}
    \usepackage{tikz}
    \usepackage[plain]{algorithm}
    \usepackage{algpseudocode}
    
    \usetikzlibrary{automata,positioning}
    
    %
    % Basic Document Settings
    %
    
    \topmargin=-0.45in
    \evensidemargin=0in
    \oddsidemargin=0in
    \textwidth=6.5in
    \textheight=9.0in
    \headsep=0.25in
    
    \linespread{1.2}
    
    \pagestyle{fancy}
    \lhead{\hmwkAuthorName}
    \chead{\hmwkClass\ \hmwkTitle}
    \rhead{\firstxmark}
    \lfoot{\lastxmark}
    \cfoot{\thepage}
    
    \renewcommand\headrulewidth{0.4pt}
    \renewcommand\footrulewidth{0.4pt}
    
    \setlength\parindent{0pt}
    \setlength\parskip{0.5cm}
    
    %
    % Create Problem Sections
    %
    
    \newcommand{\enterProblemHeader}[1]{
        \nobreak\extramarks{}{Problem \arabic{#1} continued on next page\ldots}\nobreak{}
        \nobreak\extramarks{Problem \arabic{#1} (continued)}{Problem \arabic{#1} continued on next page\ldots}\nobreak{}
    }
    
    \newcommand{\exitProblemHeader}[1]{
        \nobreak\extramarks{Problem \arabic{#1} (continued)}{Problem \arabic{#1} continued on next page\ldots}\nobreak{}
        \stepcounter{#1}
        \nobreak\extramarks{Problem \arabic{#1}}{}\nobreak{}
    }
    
    \setcounter{secnumdepth}{0}
    \newcounter{partCounter}
    \newcounter{homeworkProblemCounter}
    \setcounter{homeworkProblemCounter}{1}
    \nobreak\extramarks{Problem \arabic{homeworkProblemCounter}}{}\nobreak{}
    
    %
    % Homework Problem Environment
    %
    % This environment takes an optional argument. When given, it will adjust the
    % problem counter. This is useful for when the problems given for your
    % assignment aren't sequential. See the last 3 problems of this template for an
    % example.
    %
    \newenvironment{homeworkProblem}[1][-1]{
        \ifnum#1>0
            \setcounter{homeworkProblemCounter}{#1}
        \fi
        \section{Problem \arabic{homeworkProblemCounter}}
        \setcounter{partCounter}{1}
        \enterProblemHeader{homeworkProblemCounter}
    }{
        \exitProblemHeader{homeworkProblemCounter}
    }
    
    %
    % Homework Details
    %   - Title
    %   - Due date
    %   - Class
    %   - Author
    %
    
    \newcommand{\hmwkTitle}{Homework\ \#1}
    \newcommand{\hmwkDueDate}{February 14, 2018}
    \newcommand{\hmwkClass}{CS 4013}
    \newcommand{\hmwkAuthorName}{\textbf{Christopher Tse}}
    
    %
    % Title Page
    %
    
    \title{
        \vspace{2in}
        \textmd{\textbf{\hmwkClass:\ \hmwkTitle}}\\
        \normalsize\vspace{0.1in}\small{Due\ on\ \hmwkDueDate}\\
        \vspace{3in}
    }
    
    \author{\hmwkAuthorName}
    \date{}
    
    \renewcommand{\part}[1]{\textbf{\large Part \Alph{partCounter}}\stepcounter{partCounter}\\}
    
    %
    % Various Helper Commands
    %
    
    % Useful for algorithms
    \newcommand{\alg}[1]{\textsc{\bfseries \footnotesize #1}}
    
    % For derivatives
    \newcommand{\deriv}[1]{\frac{\mathrm{d}}{\mathrm{d}x} (#1)}
    
    % For partial derivatives
    \newcommand{\pderiv}[2]{\frac{\partial}{\partial #1} (#2)}
    
    % Integral dx
    \newcommand{\dx}{\mathrm{d}x}
    
    % Alias for the Solution section header
    \newcommand{\solution}{\textbf{\large Solution}}
    
    % Probability commands: Expectation, Variance, Covariance, Bias
    \newcommand{\E}{\mathrm{E}}
    \newcommand{\Var}{\mathrm{Var}}
    \newcommand{\Cov}{\mathrm{Cov}}
    \newcommand{\Bias}{\mathrm{Bias}}
    
    \begin{document}
    
    \maketitle
    
    \pagebreak
    
    \begin{homeworkProblem}
        Characterize the environment for each of the following tasks, using the 9 properties listed. Justify each answer, ideally with one complete sentence.
        
        \begin{enumerate}
            \item \textbf{A “smart home” that autonomously controls the thermo-stat, humidifier, air-conditioner, lights, etc.}
                \begin{itemize}
                    \item Partially observable. While it could be possible to make the system fully observable with enough sensors, realistically it will not be able to detect accurate readings for the entire home.
                    
                    \item Stochastic. There is randomness involved with changes in weather, humans affecting the environment by opening/closing doors, etc.
                    
                    \item Sequential. The actions the smart home decides to take now will affect future temperatures and environment conditions.

                    \item Dynamic. The outside weather and human actions continually affect the environment as the agent thinks.

                    \item Continuous states. We deal with real values, so there are infinitely many values each reading can measure. 

                    \item Discrete actions. For the systems listed in the question, they all have two discrete states, being either on or off. While the air conditioner can be set to a range of values, it is usually a set of discrete integer temperature values.

                    \item Continuous time. In the real world, states change with the passing of time which is continuous.

                    \item Unknown physics. As far as the system is concerned, it is not aware of the physics and thermodynamics involved with controlling a home system. It only responds with an action based on the current state and the desired state. It could be argued that it is known physics since the system knows how each action will affect the environment.
                \end{itemize}
            \item \textbf{A robotic restaurant server (e.g., waiter or waitress)}
                \begin{itemize}
                    \item Partially observable. The robot server cannot see or know everything going on in the restaurant, unless a restaurant-wide vision system was implemented using cameras mounted around the room.

                    \item Stochastic. There is randomess in the needs of the restaurant patrons, and the same person may not want the same thing every time.

                    \item Episodic. Taking care of the needs on one patron should not affect others, though it could be argued that it is sequential since serving a patron one part of their meal will make them want the next (appetizer, then entree, then possibly dessert).

                    \item Dynamic. The needs of patrons can change or increase while the agent is thinking.

                    \item Continuous states. It is the real world, so there are infinitely many places the robot server could be, but the orders and needs of the patrons could be argued to be discrete since the number of things offered by the restaurant are discrete.

                    \item Continuous actions. The robot server can move its limbs at an infinite number of ways in the real world, move at an infinite number of different velocities, etc.

                    \item Continuous time. In the real world, states change with the passing of time which is continuous.

                    \item Known physics. The robot should be aware of the physics affecting it, which means it knows how to move its limbs and/or wheels in order to perform its desired actions.
                \end{itemize}

            \item \textbf{An intelligent coach that plays fantasy hockey. The coach must play in a league against other intelligent coaches, draft its team at the beginning of the season, determine which players should start or be benched each night, and potentially make trades during the season.}
                \begin{itemize}
                    \item Fully observable. The coach can see the stats of all the players before making the decisions for who to start/bench and who to trade.

                    \item Stochastic. There is randomess in the outcome of the games which is out of control of the coach.

                    \item Sequential. Making a certain move on a certain round affects following rounds. For example, if Player A was traded, it is no longer in the coach's possession to be traded in the following turns until it receives it again from another trade.

                    \item Static. The coach thinks and executes its move every night before the games proceed.

                    \item Discrete states. The states consist of the current player roster for each coach, which is a large, but infinite, number of states.

                    \item Discrete actions. The coach decides between starting or benching a player, and whether to trade a player or not.

                    \item Discrete time. The rounds take place every night during the season, and progresses by nights.

                    \item Unknown dynamics. The coach cannot know the outcome of the game based on its decision to start/bench/trade players.
                \end{itemize}

            \item \textbf{An intelligent algorithm that predicts solar flares several days in advance. The algorithm must predict (a) whether or not a solar will occur; and (b) if so, its severity (class A, B, C, M, or X)}
                \begin{itemize}
                    \item Partially observable. If the environment is the current state of the sun, it is only possible to see approximately half of it using current tools (cannot see behind it).

                    \item Stochastic. There is randomess in the chemical reactions within the sun that cannot be determined by the agent.

                    \item Sequential. Making a certain move on a certain round affects following rounds. For example, if Player A was traded, it is no longer in the coach's possession to be traded in the following turns until it receives it again from another trade.

                    \item Static. The coach thinks and executes its move every night before the games proceed.

                    \item Discrete states. The states consist of the current player roster for each coach, which is a large, but infinite, number of states.

                    \item Discrete actions. The coach decides between starting or benching a player, and whether to trade a player or not.

                    \item Discrete time. The rounds take place every night during the season, and progresses by nights.

                    \item Unknown dynamics. The coach cannot know the outcome of the game based on its decision to start/bench/trade players.
                \end{itemize}
            \end{itemize}
        \end{enumerate}
        
        
    \end{homeworkProblem}
\end{document}